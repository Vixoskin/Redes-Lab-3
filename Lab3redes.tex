\documentclass{article}
\usepackage[utf8]{inputenc}
\usepackage{graphicx}

\title{Lab 3 Redes}
\author{Vicente Lopez\\Esteban León\\Sebastian Antón\\Jorge Ramirez\\Profesor: Jose Alejandro Perez\\Ayudante: Alexis Inzunza}
\date{Mayo 2017}

\usepackage{natbib}
\usepackage{graphicx}

\begin{document}
\begin{figure}[h]
\includegraphics[width=0.45\textwidth]{logo_udp.png}
\maketitle
\end{figure}

\section{Indice}
2 Instalacion de software ------------------------------------ 1\\
3 Creacion de paquetes -------------------------------------- 2\\
4 Cuestionario -------------------------------------------------- 2\\
5 Conclusión ---------------------------------------------------- 3\\
6 Bibliografia --------------------------------------------------- 3\\

\section{Instalacion de software:}
 Procure que los computadores en los que realizará la experiencia tengan el software necesario para operar.:\\\\
1. Inicie el computador en la distribución Linux disponible.\\
2. Inicie sesión en el equipo, recuerde que para el laboratorio de informática el usuario y contraseña es informatica y para el laboratorio de telemática el usuario y contraseña es telematica.\\
3. Inicie el terminal, en caso de no encontrarlo, presione ALT+F2 y digite “gnome-terminal”.\\
4. Instale el software necesario, digite “sudo apt-get install python-scapy”, en caso de requerir contraseña, esta fue indicada en el punto 2.\\
5. Una vez finalizado el proceso de instalación (En caso de estar instalado, aptitude se lo hará saber) inicie Wireshark con el comando “sudo scapy”.\\\\

\section{Creacion de paquetes:}
Cree una red entre los equipos que dispone el laboratorio utilizando un switch y luego de responder  todas las actividades vuelva a realizar el experimento con un hub y prepare Wireshark para capturar los  paquetes enviados y recibidos en ambos equipos.\\
Cree tres paquete en Scapy de tipo ICMP y envíelo con las siguientes opciones.\\

 1. Dirección de MAC de destino FF:FF:FF:FF:FF:FF. \\

 2. Dirección de MAC de destino de otro equipo.\\

 3. Dirección de MAC de destino que no corresponda ningún equipo de la red. \\

\section{Cuestionario:}
\subsection{¿Qué sucede cuando se envía un paquete a la dirección FF:FF:FF:FF:FF:FF? ¿Quiénes lo reciben? ¿Por qué?}
La MAC FF:FF:FF:FF:FF:FF es la notación hexadecimal de broadcast, por lo que cuando un host envía un paquete hacia esa dirección, el paquete es recibido por todos los host conectados a la red LAN.\\

\subsection{¿Qué pasa cuando se envía un paquete a la MAC de otro equipo? ¿Quiénes lo pueden recibir? ¿Por qué? }
El paquete es enviado al host con la MAC especificada ya que el paquete es de tipo Unicast (difusión única), siendo el switch quien realice la labor de buscar la MAC solicitada y posterior a ello, enviar el paquete a esa dirección. Lo unico que recibiría otro equipo, es la consulta que hace el switch para saber a quien dirigir el paquete.\\

\subsection{ ¿Qué sucede si se envía un paquete a una MAC que no corresponda a ningún equipo de la red? ¿Quiénes lo pueden recepcionar? ¿Por qué? }
Cuando se envía un paquete, el switch consulta a todos los host asociados a la red a quién pertenece la MAC a la cual se quiere enviar el paquete, por lo que al no encontrarse la MAC solicitada, el switch no envia el paquete y se le responde al equipo emisor que no se pudo realizar la entrega de este.\\

\section{Conclusion}
Gracias a este laboratorio aprendimos a como realizar un paquete y enviarlo con exito a un equipo especifico con su respectiva MAC, o bien a todos dentro de la misma LAN gracias a la dirección de broadcast.

\section{Bibliografia}
Laboratorio N3, Scapy Wireshark\\
\end{document}